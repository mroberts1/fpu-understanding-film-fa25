% Options for packages loaded elsewhere
% Options for packages loaded elsewhere
\PassOptionsToPackage{unicode}{hyperref}
\PassOptionsToPackage{hyphens}{url}
\PassOptionsToPackage{dvipsnames,svgnames,x11names}{xcolor}
%
\documentclass[
  letterpaper,
  DIV=11,
  numbers=noendperiod,
  oneside]{scrartcl}
\usepackage{xcolor}
\usepackage[left=1in,marginparwidth=2.0666666666667in,textwidth=4.1333333333333in,marginparsep=0.3in]{geometry}
\usepackage{amsmath,amssymb}
\setcounter{secnumdepth}{-\maxdimen} % remove section numbering
\usepackage{iftex}
\ifPDFTeX
  \usepackage[T1]{fontenc}
  \usepackage[utf8]{inputenc}
  \usepackage{textcomp} % provide euro and other symbols
\else % if luatex or xetex
  \usepackage{unicode-math} % this also loads fontspec
  \defaultfontfeatures{Scale=MatchLowercase}
  \defaultfontfeatures[\rmfamily]{Ligatures=TeX,Scale=1}
\fi
\usepackage{lmodern}
\ifPDFTeX\else
  % xetex/luatex font selection
\fi
% Use upquote if available, for straight quotes in verbatim environments
\IfFileExists{upquote.sty}{\usepackage{upquote}}{}
\IfFileExists{microtype.sty}{% use microtype if available
  \usepackage[]{microtype}
  \UseMicrotypeSet[protrusion]{basicmath} % disable protrusion for tt fonts
}{}
\makeatletter
\@ifundefined{KOMAClassName}{% if non-KOMA class
  \IfFileExists{parskip.sty}{%
    \usepackage{parskip}
  }{% else
    \setlength{\parindent}{0pt}
    \setlength{\parskip}{6pt plus 2pt minus 1pt}}
}{% if KOMA class
  \KOMAoptions{parskip=half}}
\makeatother
% Make \paragraph and \subparagraph free-standing
\makeatletter
\ifx\paragraph\undefined\else
  \let\oldparagraph\paragraph
  \renewcommand{\paragraph}{
    \@ifstar
      \xxxParagraphStar
      \xxxParagraphNoStar
  }
  \newcommand{\xxxParagraphStar}[1]{\oldparagraph*{#1}\mbox{}}
  \newcommand{\xxxParagraphNoStar}[1]{\oldparagraph{#1}\mbox{}}
\fi
\ifx\subparagraph\undefined\else
  \let\oldsubparagraph\subparagraph
  \renewcommand{\subparagraph}{
    \@ifstar
      \xxxSubParagraphStar
      \xxxSubParagraphNoStar
  }
  \newcommand{\xxxSubParagraphStar}[1]{\oldsubparagraph*{#1}\mbox{}}
  \newcommand{\xxxSubParagraphNoStar}[1]{\oldsubparagraph{#1}\mbox{}}
\fi
\makeatother


\usepackage{longtable,booktabs,array}
\usepackage{calc} % for calculating minipage widths
% Correct order of tables after \paragraph or \subparagraph
\usepackage{etoolbox}
\makeatletter
\patchcmd\longtable{\par}{\if@noskipsec\mbox{}\fi\par}{}{}
\makeatother
% Allow footnotes in longtable head/foot
\IfFileExists{footnotehyper.sty}{\usepackage{footnotehyper}}{\usepackage{footnote}}
\makesavenoteenv{longtable}
\usepackage{graphicx}
\makeatletter
\newsavebox\pandoc@box
\newcommand*\pandocbounded[1]{% scales image to fit in text height/width
  \sbox\pandoc@box{#1}%
  \Gscale@div\@tempa{\textheight}{\dimexpr\ht\pandoc@box+\dp\pandoc@box\relax}%
  \Gscale@div\@tempb{\linewidth}{\wd\pandoc@box}%
  \ifdim\@tempb\p@<\@tempa\p@\let\@tempa\@tempb\fi% select the smaller of both
  \ifdim\@tempa\p@<\p@\scalebox{\@tempa}{\usebox\pandoc@box}%
  \else\usebox{\pandoc@box}%
  \fi%
}
% Set default figure placement to htbp
\def\fps@figure{htbp}
\makeatother





\setlength{\emergencystretch}{3em} % prevent overfull lines

\providecommand{\tightlist}{%
  \setlength{\itemsep}{0pt}\setlength{\parskip}{0pt}}



 


% load packages
\usepackage{geometry}
\usepackage{xcolor}
\usepackage{eso-pic}
\usepackage{fancyhdr}
\usepackage{sectsty}
\usepackage{fontspec}
\usepackage{titlesec}

%% Set page size with a wider right margin
\geometry{a4paper, total={170mm,257mm}, left=20mm, top=20mm, bottom=20mm, right=50mm}

%% Let's define some colours
\definecolor{light}{HTML}{E6E6FA}
\definecolor{highlight}{HTML}{800080}
\definecolor{dark}{HTML}{330033}

%% Let's add the border on the right hand side 
% \AddToShipoutPicture{% 
%     \AtPageLowerLeft{% 
%         \put(\LenToUnit{\dimexpr\paperwidth-3cm},0){% 
%             \color{light}\rule{3cm}{\LenToUnit\paperheight}%
%           }%
%      }%
%      % logo
%     \AtPageLowerLeft{% start the bar at the bottom right of the page
%         \put(\LenToUnit{\dimexpr\paperwidth-2.25cm},27.2cm){% move it to the top right
%             \color{light}\includegraphics[width=1.5cm]{_extensions/nrennie/PrettyPDF/logo.png}
%           }%
%      }%
% }

%% Style the page number
\fancypagestyle{mystyle}{
  \fancyhf{}
  \renewcommand\headrulewidth{0pt}
  \fancyfoot[R]{\thepage}
  \fancyfootoffset{3.5cm}
}
\setlength{\footskip}{20pt}

%% style the chapter/section fonts
\chapterfont{\color{dark}\fontsize{20}{16.8}\selectfont}
\sectionfont{\color{dark}\fontsize{20}{16.8}\selectfont}
\subsectionfont{\color{dark}\fontsize{14}{16.8}\selectfont}
\titleformat{\subsection}
  {\sffamily\Large\bfseries}{\thesection}{1em}{}[{\titlerule[0.8pt]}]
  
% left align title
\makeatletter
\renewcommand{\maketitle}{\bgroup\setlength{\parindent}{0pt}
\begin{flushleft}
  {\sffamily\huge\textbf{\MakeUppercase{\@title}}} \vspace{0.3cm} \newline
  {\Large {\@subtitle}} \newline
  \@author
\end{flushleft}\egroup
}
\makeatother

%% Use some custom fonts
\setsansfont{Ubuntu}[
    Path=_extensions/nrennie/PrettyPDF/Ubuntu/,
    Scale=0.9,
    Extension = .ttf,
    UprightFont=*-Regular,
    BoldFont=*-Bold,
    ItalicFont=*-Italic,
    ]

\setmainfont{Ubuntu}[
    Path=_extensions/nrennie/PrettyPDF/Ubuntu/,
    Scale=0.9,
    Extension = .ttf,
    UprightFont=*-Regular,
    BoldFont=*-Bold,
    ItalicFont=*-Italic,
    ]
\KOMAoption{captions}{tableheading}
\makeatletter
\@ifpackageloaded{caption}{}{\usepackage{caption}}
\AtBeginDocument{%
\ifdefined\contentsname
  \renewcommand*\contentsname{Table of contents}
\else
  \newcommand\contentsname{Table of contents}
\fi
\ifdefined\listfigurename
  \renewcommand*\listfigurename{List of Figures}
\else
  \newcommand\listfigurename{List of Figures}
\fi
\ifdefined\listtablename
  \renewcommand*\listtablename{List of Tables}
\else
  \newcommand\listtablename{List of Tables}
\fi
\ifdefined\figurename
  \renewcommand*\figurename{Figure}
\else
  \newcommand\figurename{Figure}
\fi
\ifdefined\tablename
  \renewcommand*\tablename{Table}
\else
  \newcommand\tablename{Table}
\fi
}
\@ifpackageloaded{float}{}{\usepackage{float}}
\floatstyle{ruled}
\@ifundefined{c@chapter}{\newfloat{codelisting}{h}{lop}}{\newfloat{codelisting}{h}{lop}[chapter]}
\floatname{codelisting}{Listing}
\newcommand*\listoflistings{\listof{codelisting}{List of Listings}}
\makeatother
\makeatletter
\makeatother
\makeatletter
\@ifpackageloaded{caption}{}{\usepackage{caption}}
\@ifpackageloaded{subcaption}{}{\usepackage{subcaption}}
\makeatother
\makeatletter
\@ifpackageloaded{tcolorbox}{}{\usepackage[skins,breakable]{tcolorbox}}
\makeatother
\makeatletter
\@ifundefined{shadecolor}{\definecolor{shadecolor}{rgb}{.97, .97, .97}}{}
\makeatother
\makeatletter
\@ifundefined{codebgcolor}{\definecolor{codebgcolor}{named}{light}}{}
\makeatother
\makeatletter
\ifdefined\Shaded\renewenvironment{Shaded}{\begin{tcolorbox}[sharp corners, colback={codebgcolor}, boxrule=0pt, enhanced, frame hidden, breakable]}{\end{tcolorbox}}\fi
\makeatother
\makeatletter
\@ifpackageloaded{sidenotes}{}{\usepackage{sidenotes}}
\@ifpackageloaded{marginnote}{}{\usepackage{marginnote}}
\makeatother
\makeatletter
\@ifpackageloaded{fontawesome5}{}{\usepackage{fontawesome5}}
\makeatother
\usepackage{bookmark}
\IfFileExists{xurl.sty}{\usepackage{xurl}}{} % add URL line breaks if available
\urlstyle{same}
\hypersetup{
  pdftitle={COMM232 Understanding Film},
  colorlinks=true,
  linkcolor={highlight},
  filecolor={Maroon},
  citecolor={Blue},
  urlcolor={highlight},
  pdfcreator={LaTeX via pandoc}}


\title{COMM232 Understanding Film}
\author{}
\date{}
\begin{document}
\maketitle

\pagestyle{mystyle}


\pandocbounded{\includegraphics[keepaspectratio]{img/get-out.gif}}

\marginnote{\begin{footnotesize}

\href{https://www.franklinpierce.edu/index.html}{Franklin Pierce
University}\\
\href{https://www.franklinpierce.edu/academics/programs/communications.html}{Communication
Program}\\
Fall Semester 2025\\
Class meetings: Tues/Thur 10:50-12:05\\
Class location: Fitz 101\\
Instructor: Dr.~Martin Roberts\\
Office: Green Room, Fitz\\
Office hours: Thursday 09:30-10:30 a.m.\\
\href{https://franklinpierce.instructure.com/courses/21717}{Canvas}\\
\href{https://www.youtube.com/playlist?list=PLKJL7wxYKW3MGY45hmpFupwjzusAAios6}{YouTube}\\
\href{mailto:robertsm@franklinpierce.edu}{\faIcon{envelope}} \textbar{}
\href{https://github.com/mroberts1/fpu-understanding-film-fa25}{\faIcon{github}}\\

\end{footnotesize}}

This course examines film as a cultural text, focusing on aspects such
as genre, movement, and style, and elements such as scriptwriting, story
structure, character development, cinematography, editing, and sound.
Students will examine how each of these elements contributes to
influencing the viewing experience. Films screened are primarily classic
American films, but current Hollywood, experimental, or international
films may also be included. Addresses GLE Learning Outcomes Arts and
Design (AD), Oral Communication (OC), and Applied Learning (AL).
Prerequisite: GLE110.

\subsection{Reading Assignments}\label{reading-assignments}

Ed Sikov,~\emph{Film Studies: An Introduction}. Second edition. New
York: Columbia University Press, 2020. Please order this book either as
a print copy or an ebook as soon as possible.

Karen Gocsik, Dave Monahan, and Richard Barsam, \emph{Writing About
Movies}. Fifth edition. New York: W.W. Norton and Company, Inc., 2019.

Other readings will be available as PDFs via links in the syllabus.
Please download all PDFs, print them out and mark them up when reading.

\subsection{Assignments \& Evaluation}\label{assignments-evaluation}

Homework exercises (based on Sikov, Film Studies) (4, biweekly): 50\%\\
Midterm (take-home): 15\%\\
Final (take-home): 15\%\\
Engagement (including attendance): 20\%

\subsection{Class Schedule}\label{class-schedule}

\emph{Week 1}

08/26 Introduction

\marginnote{\begin{footnotesize}

\url{https://vimeo.com/128097765}

\end{footnotesize}}

08/28~~\textbf{What Is (a) Film?}

\begin{itemize}
\tightlist
\item
  \emph{The Definition of Film}~(Richard Misek, 2015)
\end{itemize}

\begin{center}\rule{0.5\linewidth}{0.5pt}\end{center}

\emph{Week 2}

09/02~\textbf{What Is Cinema?}

\marginnote{\begin{footnotesize}

\url{https://youtu.be/9oU*myNlk1A}

\url{https://youtu.be/289zm-hlMWk}

\url{https://youtu.be/Y4EFuZxEtNI}

\end{footnotesize}}

André Bazin, ``\href{pdf/bazin-ontology.pdf}{Ontology of the
Photographic Image}''

09/04~\textbf{The Cinematic}

\begin{itemize}
\item
  \emph{Detectives} (Apple, 2022)
\item
  \emph{Whodunnit} (Apple, 2022)
\item
  \emph{Why Do A24 Films Look Like That?}
\item
  Matt Zoller Seitz and Chris
  Wade,~``\href{https://www.vulture.com/2015/10/cinematic-tv-what-does-it-really-mean.html}{What
  Does `Cinematic TV' Really Mean?}'' (\emph{Vulture}, 21 October 2015)
\item
  Gocsik, Monahan, and Barsam, ``The Challenges of Writing About
  Movies'' (\emph{Writing About Movies}, ch.~1)
\end{itemize}

\begin{center}\rule{0.5\linewidth}{0.5pt}\end{center}

\emph{Week 3}

09/09 ~\textbf{What is Film Analysis?}

\begin{itemize}
\tightlist
\item
  Norman N.
  Holland,~\href{https://www.asharperfocus.com/Seeing-Movies-Now.html}{``Seeing
  Movies Today---Alas}\\
\item
  Norman N.
  Holland,~\href{https://www.asharperfocus.com/criticism.html}{``Criticism
  vs.~Reviewing''}
\end{itemize}

09/11

\begin{itemize}
\item
  Sikov, ``\href{pdf/sikov-mise-en-scene-1-2.pdf}{Mise-en-scène}''
  (\emph{Film Studies}, chs.~1-2)
\item
  \href{https://kogonada.com/archive}{Kogonada video essays}
\item
  Gocsik, Monahan, and Barsam, ``Looking at Movies'' (\emph{Writing
  About Movies}, ch.~2)
\end{itemize}

\begin{center}\rule{0.5\linewidth}{0.5pt}\end{center}

\emph{Week 4}

09/16

NO CLASS (Instructor Absent)

\begin{itemize}
\tightlist
\item
  Sikov, ``\href{pdf/sikov-montage.pdf}{Montage}'' (\emph{Film Studies},
  ch.~4)
\end{itemize}

09/18~~\textbf{Kino-eye}

\begin{itemize}
\item
  \href{pdf/kino-eye.pdf}{Kino-eye readings}~~(Krakauer, Michelson,
  Vertov)
\item
  Watch:~\emph{The History of Cutting}, 15 mins.
\item
  \href{https://franklinpierce.on.worldcat.org/oclc/52408086}{\emph{Man
  With A Movie Camera}}~(Dziga Vertov, 1929) {[}on DVD at the library{]}
\item
  Gocsik, Monahan, and Barsam, ``Formal Analysis'' (\emph{Writing About
  Movies}, ch.~3)
\end{itemize}

\begin{center}\rule{0.5\linewidth}{0.5pt}\end{center}

\emph{Week 5}

09/23

\begin{itemize}
\tightlist
\item
  Sikov,~\emph{Film Studies}, ch.~3 (``Cinematography'')
\end{itemize}

09/25~

\textbf{Hollywood}

\begin{itemize}
\tightlist
\item
  Dixon and Foster, ``\href{pdf/dixon-foster-ch4.pdf}{The Hollywood
  Studio System in the 1930s and 1940s}'' (\emph{A Short History of
  Film}, chapter 4)
\item
  Farran Smith Nehme,
  ``\href{https://www.criterion.com/current/posts/6331-george-cukor-s-way-with-women}{George
  Cukor's Way With Women}'' (Criterion, 2 May 2019)
\end{itemize}

In-class: What~\emph{Price Hollywood?}~(George Cukor, 1932)
{[}excerpt{]}

\begin{itemize}
\tightlist
\item
  Gocsik, Monahan, and Barsam, ``Cultural Analysis'' (\emph{Writing
  About Movies}, ch.~4)
\end{itemize}

\textbf{DEADLINE: Homework 2: Opening Sequence}

\begin{center}\rule{0.5\linewidth}{0.5pt}\end{center}

\emph{Week 6}

09/30~\textbf{Color}

\begin{itemize}
\item
  Richard Misek, ``\href{pdf/Film-Color.pdf}{Film Color}''
  (in~\emph{Chromatic Cinema: A History of Screen Color}~(Chichester:
  Wiley-Blackwell, 2010), ch.~1)
\item
  Video
  essay:~\href{https://www.criterionchannel.com/color-motifs-in-black-narcissus}{Color~\emph{Motifs
  in 'Black Narcissus}}~(Kristin Thompson, 2018) {[}Criterion Channel -
  requires subscription{]}
\end{itemize}

10/02 ~

\begin{itemize}
\item
  Maria Helena Braga e Vaz da Costa,
  ``\href{pdf/Color-in-Films.pdf}{Color in Films: A Critical Overview}''
\item
  \href{https://filmcolors.org/}{Timeline of Historical Film Colors}
\item
  Gocsik, Monahan, and Barsam, ``Generating Ideas'' (\emph{Writing About
  Movies}, ch.~5)
\end{itemize}

\begin{center}\rule{0.5\linewidth}{0.5pt}\end{center}

\emph{Week 7}

10/07 ~\textbf{Narrative}

\begin{itemize}
\tightlist
\item
  Sikov,~\emph{Film Studies}, ch.~6 (``Narrative: From Scene to Scene'')
\end{itemize}

10/09~

\begin{itemize}
\item
  Sikov, ch.~7 (``From Screenplay to Film'')
\item
  ``\href{https://en.wikipedia.org/wiki/Fabula*and*syuzhet}{Fabula and
  Syuzhet}'' (\emph{Wikipedia})
\item
  Gocsik, Monahan, and Barsam, ``Researching Movies'' (\emph{Writing
  About Movies}, ch.~6)
\end{itemize}

\begin{center}\rule{0.5\linewidth}{0.5pt}\end{center}

\emph{Week 8}

10/14 NO CLASS (Fall Break)

10/16~\textbf{Midterm Quiz}

\href{https://franklinpierce.instructure.com/courses/19450/pages/midterm-quiz-study-guide}{Study
Guide}

\href{https://franklinpierce.instructure.com/courses/19450/pages/midterm-quiz-fall-2023}{Example:
last year's quiz}

\begin{center}\rule{0.5\linewidth}{0.5pt}\end{center}

\emph{Week 9}

10/21 \textbf{Auteurism and the New Wave}

\begin{itemize}
\tightlist
\item
  Sikov,~\emph{Film Studies}, ch.~8 (``Filmmakers'')
\item
  Alexandre Astruc,
  ``\href{http://www.newwavefilm.com/about/camera-stylo-astruc.shtml}{The
  Birth of a New Avant-Garde: La Caméra-Stylo}''
\end{itemize}

François Truffaut,
``\href{http://www.newwavefilm.com/about/a-certain-tendency-of-french-cinema-truffaut.shtml}{A
Certain Tendency of French Cinema}''

10/23~

\begin{itemize}
\tightlist
\item
  \href{https://cinemawavesblog.com/movements/french-new-wave/}{French
  New Wave}~(CinemaWaves website)
\item
  Samuel
  Harries,~\href{https://www.movementsinfilm.com/french-new-wave}{Movements
  in Film: French New Wave}
\item
  \href{http://www.newwavefilm.com/new-wave-cinema-guide/nouvelle-vague-where-to-start.shtml}{French
  New Wave: Where to Start}
\end{itemize}

Recommended films:

\begin{itemize}
\tightlist
\item
  \emph{The 400 Blows}~(François Truffaut, 1959)\\
\item
  \emph{Breathless}~(Jean-Luc Godard, 1960)\\
\item
  \emph{Cleo from 5 to 7}~(Agnes Varda, 1961)
\end{itemize}

Gocsik, Monahan, and Barsam, ``Developing Your Thesis'' (\emph{Writing
About Movies}, ch.~7)

\begin{center}\rule{0.5\linewidth}{0.5pt}\end{center}

\emph{Week 10}

10/28~\textbf{Soundtracks}

\begin{itemize}
\tightlist
\item
  Sikov, ``Sound'' (Film Studies, ch.~5)
\end{itemize}

10/30~

\href{https://markkorven.com/}{Mark Korven}~YouTube videos\\
\href{https://vimeo.com/user3424223}{Mark Korven Vimeo}

Watch:~\emph{The Witch}~(Robert Eggers, 2015)

\textbf{DEADLINE: Homework 3: Editing}

\begin{itemize}
\tightlist
\item
  Gocsik, Monahan, and Barsam, ``Attending to Style'' (\emph{Writing
  About Movies}, ch.~8)
\end{itemize}

\begin{center}\rule{0.5\linewidth}{0.5pt}\end{center}

\emph{Week 11}

11/04~

11/06~\textbf{Art}

\begin{itemize}
\item
  David Bordwell, ``\href{pdf/bordwell-art-cinema.pdf}{The Art Cinema as
  a Mode of Film Practice}
\item
  Gocsik, Monahan, and Barsam, ``Considering Structure and
  Organization'' (\emph{Writing About Movies}, ch.~9)
\end{itemize}

In-class:~\emph{La Notte}~(Michelangelo Antonioni, 1962) (excerpt)

\begin{center}\rule{0.5\linewidth}{0.5pt}\end{center}

\emph{Week 12}

11/11~~\textbf{Genre}

\begin{itemize}
\tightlist
\item
  Sikov,~\emph{Film Studies}, ch.~10 (``Genre'')
\item
  Rick Altman, ``\href{pdf/altman-genre.pdf\%22}{A Semantic/Syntactic
  Approach to Film Genre}
\end{itemize}

11/13~

\begin{itemize}
\item
  Ginette Vincendeau,
  ``\href{https://www2.bfi.org.uk/news-opinion/sight-sound-magazine/features/deep-focus/french-film-noir}{How
  The French Birthed Film Noir}''
\item
  Gocsik, Monahan, and Barsam, ``Revising Your Work (\emph{Writing About
  Movies}), ch.~10)
\end{itemize}

\begin{center}\rule{0.5\linewidth}{0.5pt}\end{center}

\emph{Week 13}

11/18 ~~\textbf{Digital Cinema}

Sikov,~\emph{Film Studies}, ch.~13 (``Film Studies in the Era of Digital
Cinema'')

11/20 ~

\begin{itemize}
\tightlist
\item
  Graham Edwards,
  ``\href{https://franklinpierce.instructure.com/courses/19450/files/2982722?wrap=1}{Explosive
  Cocktail}~\href{https://franklinpierce.instructure.com/courses/19450/files/2982722/download?download*frd=1}{Download
  Explosive Cocktail}'' (\emph{Cinefex}~145 (February 2016): 12-39 ~-
  Prince,~\emph{Digital Cinema}~(selected chs.)
\end{itemize}

\textbf{DEADLINE: Homework 4: Genre}

\begin{center}\rule{0.5\linewidth}{0.5pt}\end{center}

\emph{Week 14}

11/25

11/27 NO CLASS (Thanksgiving)

\begin{center}\rule{0.5\linewidth}{0.5pt}\end{center}

\emph{Week 15}

12/02~

12/04~Conclusion

\begin{center}\rule{0.5\linewidth}{0.5pt}\end{center}

12/05 Classes End

12/06-07 Reading Days

12/08-11 Final Exam Period

\begin{center}\rule{0.5\linewidth}{0.5pt}\end{center}

\subsection{Policies \& Procedures}\label{policies-procedures}

\subsubsection{How to do well in this
class}\label{how-to-do-well-in-this-class}

Attend all scheduled classes.

Be on time. Don't leave early.

Complete all reading assignments \textbf{before} the scheduled class.

Watch all assigned films.

Take notes. The Cornell note-taking system is recommended. If you don't
take notes during class, you will forget 90\% of what was discussed
there.~

Use a free note-taking app
like~\href{https://obsidian.md/}{Obsidian}~and learn how to write in
Markdown format.~

Ask questions. These should be academic (i.e.~related to course topics
and materials) rather than administrative (due dates, rubrics etc.).

Administrative questions should be asked outside class, either in person
after class or via Canvas Inbox.

\subsubsection{Attendance}\label{attendance}

Attendance of all scheduled classes is mandatory. Attendance is taken at
the beginning of class; if you arrive late, you may miss rolecall, in
which case remember to ask to be recorded as Late after class.

\subsubsection{Punctuality}\label{punctuality}

Class begins at 10:50 a.m. and ends at 12:05 p.m. Be sure to arrive in
time for the beginning of class. Class does not end until students are
dismissed: please do not leave early. Arrival after 11 a.m. will be
recorded as Late (L), and three Late count as one Absence. If you need
to leave early, please notify me of this before class as a courtesy.

\subsubsection{Absences}\label{absences}

You are permitted up to two (2) Excused absences, and a maximum of four
(4) absences for any reason.

If you reach 4 absences, you are required to arrange a meeting with me
to discuss your status in the course.

\subsubsection{Excused/Unexcused
Absence}\label{excusedunexcused-absence}

Absences are recorded as Excused (E) in case of health, emergency,
bereavement, professional development (e.g.~job interview), or
participation in a University sports team. For more than two (2)
absences in case of health issues, a doctor's note is required. All
other absences are recorded as Unexcused (U).

\subsubsection{Sports Teams}\label{sports-teams}

If you are unable to attend a class due to attending sports events, you
are required to email me a list of dates when you expect to be absent.
After the missed class, you are required to schedule a meeting with me
during my office hour (Thursday 9:30-10:30) to discuss what was missed
during your absence.

\subsubsection{Communication}\label{communication}

Please use Canvas Inbox for all course-related questions. Other
questions can be sent via email
(\href{mailto:robertsm@franklinpierce.edu}{\nolinkurl{robertsm@franklinpierce.edu}}).~I
try to reply to direct messages within 24 hours (except on weekends),
and will expect you to do the same.

\subsubsection{Notifications}\label{notifications}

It is strongly recommended to disable notifications on Canvas.~

\subsection{Classroom Protocol}\label{classroom-protocol}

\subsubsection{Food and Drink}\label{food-and-drink}

You may bring a drink to class, but please do not eat when class is in
progress.

\subsubsection{Sleeping in class}\label{sleeping-in-class}

You are expected to remain alert and attentive at all times during
class. If you fall asleep during class, you will receive an informal
verbal warning. If the problem recurs, you will receive a second written
warning. If the problem continues after this, it will significantly
affect your Engagement grade for the course (20\%).~

\subsubsection{Cellphones}\label{cellphones}

The use of cellphones during class is strictly forbidden.

\subsubsection{Headphones}\label{headphones}

The use of headphones or earbuds during class is strictly forbidden~

\subsubsection{Laptop Privileges}\label{laptop-privileges}

The use of laptops or tablets (e.g.~iPads) is permitted during class for
taking notes ONLY. Any other use is unauthorized, and if detected will
result in the withdrawal of your laptop privileges for the remainder of
the semester. This means that you will be required to take notes by hand
with pen and paper from that point on.

\subsubsection{Timeouts}\label{timeouts}

Since this is a short class (1.25 hrs), you are expected to remain
present in class throughout except in case of emergency. Please do not
keep walking in and out of class. This applies particularly when
screenings are in progress.

\subsubsection{Hoods}\label{hoods}

Hoods must be worn~\textbf{down}~at all times. This is so that I can see
at all times that you are not using headphones (see above). Room
temperature does not require wearing of hats or other headgear.

\subsubsection{Warnings and Engagement
Grade}\label{warnings-and-engagement-grade}

Non-compliance with the above policies will result in an initial verbal
warning.

If the non-compliance continues, I will request a meeting with you to
discuss the issue, and it will impact your Engagement grade (20\%).

\begin{center}\rule{0.5\linewidth}{0.5pt}\end{center}




\end{document}
